\documentclass[a4paper, 12pt]{article}

%Modelo baseados nos diponibilizados por Lazaro Silva (https://www.overleaf.com/latex/templates/modelo-de-relatorio/dnxhwsjdmpyb)

\usepackage[portuges]{babel}
\usepackage[utf8]{inputenc}
\usepackage{amsmath}
\usepackage{indentfirst}
\usepackage{graphicx}
\usepackage{multicol,lipsum}
\usepackage{multirow}
\usepackage[table,xcdraw]{xcolor}
\usepackage{url}

\begin{document}
%\maketitle

\begin{titlepage}
	\begin{center}

 	\graphicspath{{imagens/}}
	\begin{figure}[!h]
		\centering
		\includegraphics[scale=0.2]{unifesp}
		\label{fig:logo_unifesp}
	\end{figure}

		\normalsize{Universidade Federal de São Paulo}\\
		\normalsize{Instituto de Ciência e Tecnologia}\\ 
		\normalsize{Arquitetura e Organização de Computadores - Turma IB}\\ 
		\vspace{15pt}
        \vspace{95pt}
        \textbf{\LARGE{Sistema de apoio à doação de órgãos implementado em Assembly}}\\
		\vspace{3,5cm}
	\end{center}
	
	\begin{flushleft}
		\begin{tabbing}
			\textbf{Integrantes:} \\
            João Vítor Moreira Gomes - RA 176536 \\
            David Souza Santos - RA 176235 \\
            Daniel Monteiro Ribeiro - RA 176231 \\
            Thiago Sole Gomes Heleno - RA 176579 \\
			\textbf{Docente responsável:} Profa. Dra. Thaína A. A. Tosta\\
	\end{tabbing}
 \end{flushleft}
	\vspace{1cm}
	
	\begin{center}
		\vspace{\fill}
			 São José dos Campos\\
    		 2025/2
			\end{center}
\end{titlepage}
%%%%%%%%%%%%%%%%%%%%%%%%%%%%%%%%%%%%%%%%%%%%%%%%%%%%%%%%%%%

% % % % % % % % %FOLHA DE ROSTO % % % % % % % % % %

\begin{titlepage}
	\begin{center}

 	\graphicspath{{imagens/}}
	\begin{figure}[!h]
		\centering
		\includegraphics[scale=0.2]{unifesp}
		\label{fig:logo_unifesp}
	\end{figure}

		\normalsize{Universidade Federal de São Paulo}\\
		\normalsize{Instituto de Ciência e Tecnologia}\\ 
		\normalsize{Arquitetura e Organização de Computadores - Turma IB}\\ 
		\vspace{15pt}
        \vspace{95pt}
        \textbf{\LARGE{Sistema de apoio à doação de órgãos implementado em Assembly}}\\
		%\title{{\large{Título}}}
		\vspace{3,5cm}
	\end{center}
\vspace{1,5cm}
	
	\begin{flushright}

   \begin{list}{}{
      \setlength{\leftmargin}{4.5cm}
      \setlength{\rightmargin}{0cm}
      \setlength{\labelwidth}{0pt}
      \setlength{\labelsep}{\leftmargin}}

      \item Segundo relatório do projeto de Arquitetura e Organização de Computadores (Turma IB) da Universidade Federal de São Paulo para cumprimento dos requisitos de avaliação e aprovação na disciplina.

   \end{list}
\end{flushright}
\vspace{1cm}
\begin{center}
		\vspace{\fill}
		 São José dos Campos\\
		 2025/2
			\end{center}
\end{titlepage}
\newpage
% % % % % % % % % % % % % % % % % % % % % % % % % %
%\newpage
%\tableofcontents
%\thispagestyle{empty}

\newpage
\pagenumbering{arabic}
% % % % % % % % % % % % % % % % % % % % % % % % % % %
\begin{abstract}
    Resumo atualizado conforme sugestões de entregas anteriores.
\end{abstract}

\newpage
\section{Introdução}
Esta seção corresponde à ``Contextualização'' das entregas anteriores. É preciso que todas as informações citadas nesta seção tenham referências, com citação por \cite{landc} e edição do arquivo .bib deste projeto latex. A escrita a ser incluída aqui deve conter a aplicação do sistema proposto (possivelmente em múltiplos parárgafos), com o tema da notícia, e ter um parágrafo final descrevendo brevemente o que o sistema propõe, baseado nas informações anteriormente citadas.

\subsection{Objetivos}

O objetivo geral deste projeto é desenvolver um sistema que ofereça apoio às equipes das organizações responsáveis pela mediação entre doadores, hospitais e receptores. O sistema visa automatizar etapas do processo de identificação e comunicação sobre a disponibilidade de órgãos, reduzindo esforços manuais e tornando o fluxo mais rápido e eficiente.

Os objetivos específicos do trabalho são:

\begin{itemize}
    \item Permitir o cadastro de potenciais doadores, registrando órgãos disponíveis no sistema;
    \item Facilitar a consulta de disponibilidade de órgãos por parte de receptores e equipes médicas;
    \item Oferecer uma visão clara da rede de hospitais e centros de distribuição, permitindo localizar rapidamente onde cada órgão está disponível;
    \item Reduzir o tempo de resposta no processo de busca e confirmação de órgãos, automatizando etapas que antes eram manuais;
    \item Apoiar a tomada de decisões das equipes responsáveis, fornecendo informações organizadas e atualizadas em um único sistema.
\end{itemize}


\section{Materiais e métodos}
Esta seção deve descrever os dados usados na implementação (materiais) e as funções e implementações realizadas (métodos). Para isso, é possível criar subseções como (fiquem livres para editá-las de acordo com o trabalho de cada grupo):

\subsection{Bases de dados}
Descrever as bases de dados usadas para compor o sistema. Por exemplo, caso a proposta contenha comparações de dados com informações diversas obtidas por referência confiáveis, citar essas informações aqui. Caso contrário, é possível remover essa subseção.

\subsection{Implementação}
Listar as operações já implementadas para o sistema. É importante relacionar as implementações com os objetivos listados, para que o trabalho faça sentido. Apresentar apenas referência ao nomes das funções usadas para essas implementações, não sendo necessário adicionar o código completo.

\subsection{Estrutura de grafo}

Para representar as relações entre possíveis doadores e centros de distribuição de órgãos, o sistema utiliza uma estrutura de grafo ponderado e não direcionado. A escolha por um grafo se justifica pela necessidade de realizar buscas eficientes baseadas em distância, critério fundamental para localizar rapidamente onde determinado órgão está disponível e para apoiar a tomada de decisões envolvendo a escolha do centro de distribuição mais próximo ou do doador mais acessível. Como cada aresta possui um peso correspondente à distância em quilômetros, essa modelagem permite empregar algoritmos de busca adequados para caminhos mínimos, automatizando etapas que antes dependeriam de consulta manual e reduzindo significativamente o tempo de resposta na identificação e confirmação de disponibilidade de órgãos. Assim, o uso de grafos contribui diretamente para tornar o processo mais ágil, organizado e eficiente, alinhando-se aos propósitos centrais do sistema.

O grafo é composto por dois tipos distintos de vértices: centros de distribuição e possíveis doadores. Ambos os tipos armazenam informações estruturadas, incluindo:
\begin{itemize}
    \item \textbf{ID}: número inteiro utilizado como índice do vértice no grafo;
    \item \textbf{Tipo};
    \item \textbf{Nome};
    \item \textbf{Cidade};
    \item \textbf{Lista de órgãos disponíveis}.
\end{itemize}

Os centros de distribuição representam hospitais ou unidades especializadas responsáveis pelo armazenamento e gestão dos órgãos disponíveis. Já os vértices do tipo doador representam indivíduos registrados no sistema, que podem possuir uma lista de órgãos doados (possivelmente vazia).

As arestas são estabelecidas exclusivamente entre doadores e centros de distribuição e entre cada centro de distribuição, representando a distância física entre esses pontos. Não são criadas conexões entre dois doadores, uma vez que tais relações não contribuem para os objetivos do sistema. Essa modelagem garante que o grafo reflita de maneira precisa o fluxo relevante da aplicação: a busca por um centro adequado a partir de um doador registrado ou a identificação de doadores próximos a um centro requisitante.

\subsubsection{Implementação do grafo}

A representação escolhida para o grafo foi a \textbf{matriz de adjacência}. Embora não seja a estrutura mais eficiente em termos de ocupação de memória — especialmente por resultar em uma matriz esparsa — ela foi adotada por ser a alternativa mais simples de implementar no contexto da linguagem Assembly utilizada. Uma implementação por lista de adjacência exigiria também a construção de listas encadeadas, o que aumentaria significativamente a complexidade estrutural e tornaria o desenvolvimento inviável para os prazos e ferramentas do projeto.

Na matriz de adjacência, cada posição $(u, v)$ armazena diretamente o peso da aresta entre os vértices $u$ e $v$. Como o grafo é não direcionado, a matriz é simétrica. Para pares de vértices sem conexão, atribui-se o valor \textbf{0}. Esse valor não gera ambiguidades, pois não existem distâncias reais iguais a zero entre os nós do sistema.

A implementação do grafo em Assembly utiliza duas funções principais:
\begin{itemize}
    \item \textbf{init\_graph}: responsável por inicializar toda a matriz de adjacência com zeros e, em seguida, carregar no grafo todas as arestas lidas do arquivo de entrada;
    \item \textbf{add\_edge}: recebe dois vértices e um peso, inserindo a aresta correspondente na matriz de adjacência (em ambas as posições $(u,v)$ e $(v,u)$).
\end{itemize}

Como o grafo não é dinâmico, seu conteúdo não se altera após a inicialização. Ele é definido integralmente com base no arquivo de texto fornecido pelo usuário, seguindo o padrão descrito na subseção de entrada de dados. O grafo somente mudará caso o arquivo seja modificado ou substituído. Essa abordagem garante simplicidade e previsibilidade na manipulação das estruturas durante as operações de busca.


\section{Resultados}
Listar testes que podem ser executados e estão funcionais no código enviado. Descrever entrada, operação e saída de cada um deles. Sugestão de adicionar prints das telas para verificação na apresentação (isso não é recomendado em outros trabalhos, como TCC, mas no nosso caso é uma boa ferramenta para facilitar a compreensão do sistema implementado).

\section{Planejamento e cronograma de execução}
Adicionar o cronograma final que foi seguido, identificando quem fez determinada parte. Não é necessário identificar quando foi feito, a não ser que queiram pela tabela abaixo. Um opção é listar nomes vs atividades pelo comando \begin{itemize}     \item      \end{itemize}.

\begin{table}[!htpb]
\centering
\begin{tabular}{|l|l|llll|}
\hline
\multicolumn{1}{|c|}{}                                     & \multicolumn{1}{c|}{}                                       & \multicolumn{4}{c|}{\textbf{Meses}}                                                                                                                                                \\ \cline{3-6} 
\multicolumn{1}{|c|}{\multirow{-2}{*}{\textbf{Atividade}}} & \multicolumn{1}{c|}{\multirow{-2}{*}{\textbf{Responsável}}} & \multicolumn{1}{c|}{\textbf{Nov.}}            & \multicolumn{1}{c|}{\textbf{Dez.}}            & \multicolumn{1}{c|}{\textbf{Jan.}}            & \multicolumn{1}{c|}{\textbf{Fev.}} \\ \hline
1                                                          & Integrante 1                                                & \multicolumn{1}{l|}{\cellcolor[HTML]{C0C0C0}} & \multicolumn{1}{l|}{}                         & \multicolumn{1}{l|}{}                         &                                    \\ \hline
2                                                          & Integrante 2                                                & \multicolumn{1}{l|}{}                         & \multicolumn{1}{l|}{\cellcolor[HTML]{C0C0C0}} & \multicolumn{1}{l|}{\cellcolor[HTML]{C0C0C0}} &                                    \\ \hline
3                                                          & Integrante 3                                                & \multicolumn{1}{l|}{}                         & \multicolumn{1}{l|}{\cellcolor[HTML]{C0C0C0}} & \multicolumn{1}{l|}{\cellcolor[HTML]{C0C0C0}} &                                    \\ \hline
4                                                          & Integrante 1                                                & \multicolumn{1}{l|}{}                         & \multicolumn{1}{l|}{\cellcolor[HTML]{C0C0C0}} & \multicolumn{1}{l|}{\cellcolor[HTML]{C0C0C0}} &                                    \\ \hline
5                                                          & Integrante 2                                                & \multicolumn{1}{l|}{}                         & \multicolumn{1}{l|}{}                         & \multicolumn{1}{l|}{\cellcolor[HTML]{C0C0C0}} & \cellcolor[HTML]{C0C0C0}           \\ \hline
6                                                          & Integrante 3                                                & \multicolumn{1}{l|}{}                         & \multicolumn{1}{l|}{}                         & \multicolumn{1}{l|}{}                         & \cellcolor[HTML]{C0C0C0}           \\ \hline
\end{tabular}
\end{table}

Além de meses, podem ser considerados períodos de tempo diversos, como períodos quinzenais. Ferramentas online podem ser usadas para isso (\url{https://www.tablesgenerator.com/}).

\section{Conclusão}
Escrever um parágrafo descrevendo o contexto do problema, e as funcionalidades do sistema implementado. Adicionar parágrafos adicionais com as limitações do sistema e trabalhos futuros (o que mais poderia ser feito caso tivéssemos tempo hábil para isso?).

\bibliographystyle{ieeetr} 
\bibliography{bibliography}
 
%\newpage
%\addcontentsline{toc}{section}{Anexo}
%\section*{Anexo}
\end{document}